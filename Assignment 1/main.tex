\documentclass[12pt]{article}
\usepackage[paper=letterpaper,margin=2cm]{geometry}
\usepackage{amsmath}
\usepackage{amssymb}
\usepackage{amsfonts}
\usepackage{newtxtext, newtxmath}
\usepackage{enumitem}
\usepackage{titling}
\usepackage[colorlinks=true]{hyperref}

\setlength{\droptitle}{-6em}

% Enter the specific assignment number and topic of that assignment below, and replace "Your Name" with your actual name.
\title{CS-E4850 Computer Vision
\\Exercise Round 1}
\author{Amirreza Akbari}
\date{\today}

\begin{document}
\maketitle
\begin{enumerate}[leftmargin=\labelsep]
\item Homogeneous coordinates.
    \begin{enumerate}
    \item The equation of a line in the plane is
    \begin{equation}\label{equ:points-on-the-line}
        ax + by + c = 0
    \end{equation}
    Show that by using homogeneous coordinates this can be written as
    \begin{equation}
        x^Tl = 0
    \end{equation}
    where $l = (a,\,b,\,c)^T$.
    
    \textbf{Solution:} 
    We are talking about points lies on the line in the plane. As far as we know, a point lies on the line $(a,\,b,\,c)$ if and only if $ax+by+c = 0$. This point $(x,\,y)$ in $\mathbb{R}^2$ can be also represent in $\mathbb{R}^3$ by adding a final coordinate $1$. We can also represent the Equation \ref{equ:points-on-the-line} as inner product of two vectors $(x,\,y,\,1)$ and $(a,\,b,\,c)^T$. To consider the set of vectors $(kx,\,ky,\,k)^T$ for varying values of $k$ to be a representation of the point $(x,\,y)^T$ in $\mathbb{R}^2$, because $(kx,\,ky,\,k)l = 0$ if and only if $(x,\,y,\,1)l = 0$. Thus, just as with lines, points are represented by homogeneous vectors. An arbitrary homogeneous vector representative of a point is of the form $x = (x_1, x_2, x_3)^T$, representing the point $(\frac{x1}{x3}, \frac{x2}{x3})^T$ in $\mathbb{R}^2$. Points, then, as homogeneous vectors are also elements of $\mathbb{P}^2$.
    
    \item  Show that the intersection of two lines $l$ and $l'$ is the point $x = l \times l'$.
    
    \textbf{Solution:} It is clear that for intersection we need a point $x$ that satisfies in both equations $x^Tl = 0$ and $x^Tl' = 0$. Let $x = l \times l'$. It is clear that due to the hint, $l^T(l\times l') = {l'}^T(l \times l') = 0$, because two of vectors in the scalar triple product are parallel. Therefore, $x^T l = x^T l' = 0$, so $x$ satisfies both equations.
    \item  Show that the line through two points $x$ and $x'$ is $l = x \times x'$.
    
    \textbf{Solution:}
    Again, we need a line $l$ that holds the following equation:
    \begin{equation}\label{equ:two-point-on-a-line}
        x^Tl = x^Tl' = 0
    \end{equation}
    Consider $l = (x \times x')$, due to the scalar triple product hint, $x^T(x\times x') = x'^T(x\times x') = 0 $, so $(x \times x')$ can be a line that satisfies the Equation \ref{equ:two-point-on-a-line}.
    \item  Show that for all $\alpha \in \mathbb{R}$ the point $y = \alpha x + (1 - \alpha)x'$ lies on the line through points $x$ and $x'$.
    
    \textbf{Solution:} According to previous part, $l = x \times x'$. As we know inner product is a linear function, so $y^T l = \alpha x^Tl + (1-\alpha) x'^Tl = 0 + 0$; so $y$ is also on this line.
    \end{enumerate}

\item  Transformations in 2D
    \begin{enumerate}
    \item  Use homogeneous coordinates and give the matrix representations of the following transformation groups: translation, Euclidean transformation (rotation+translation), similarity transformation (scaling+rotation+translation), affine transformation, projective transformation.
    
    \textbf{Solution:}
    \begin{itemize}
        \item Translation:
        \begin{equation}
            \begin{pmatrix}
            x+t_x \\
            y+t_y \\
            1 
            \end{pmatrix}= 
            \begin{pmatrix}
            1 & 0 &  t_x \\
            0 & 1 & t_y \\
            0 & 0 & 1
            \end{pmatrix}
            \begin{pmatrix}
            x \\
            y \\
            1 
            \end{pmatrix}
        \end{equation}
        \item Euclidean:
            \begin{equation}
                \begin{pmatrix}
                1& 0 &  t_x \\
                0 & 1 & t_y \\
                0 & 0 & 1
                \end{pmatrix}
                \begin{pmatrix}
                    cos(\theta) & -sin(\theta) & 0 \\
                    sin(\theta) & cos(\theta) & 0 \\
                    0 & 0 & 1
                \end{pmatrix}= 
                \begin{pmatrix}
                    cos(\theta) & -sin(\theta) &  t_x \\
                    sin(\theta) & cos(\theta) & t_y \\
                    0 & 0 & 1
                \end{pmatrix}
            \end{equation}
            \begin{equation}
            \Rightarrow
            \begin{pmatrix}
                cos(\theta) & -sin(\theta) &  t_x \\
                sin(\theta) & cos(\theta) & t_y \\
                0 & 0 & 1
                \end{pmatrix}
                \begin{pmatrix}
                    x \\
                    y \\
                    1 
                \end{pmatrix}
                =
                    \begin{pmatrix}
                    x' \\
                    y' \\
                    1 
                    \end{pmatrix}
            \end{equation}
        \item Similarity:
        \begin{equation}
        \begin{pmatrix}
    1 & 0 &  t_x \\
    0 & 1 & t_y \\
0 & 0 & 1
\end{pmatrix}
\begin{pmatrix}
a & 0 &  0 \\
0 & a & 0 \\
0 & 0 & 1
\end{pmatrix}
\begin{pmatrix}
cos(\theta) & -sin(\theta) &  t_x \\
sin(\theta) & cos(\theta) & t_y \\
0 & 0 & 1
\end{pmatrix}
=
\begin{pmatrix}
a\ cos(\theta) & -a\ sin(\theta) &  t_x \\
a\ sin(\theta) & a\ cos(\theta) & t_y \\
0 & 0 & 1
\end{pmatrix}
\end{equation}

    \begin{equation}
    \Rightarrow 
\begin{pmatrix}
a\ cos(\theta) & -a\ sin(\theta) &  t_x \\
a\ sin(\theta) & a\ cos(\theta) & t_y \\
0 & 0 & 1
\end{pmatrix}
\begin{pmatrix}
        x \\
        y \\
        1 
        \end{pmatrix}
        =
        \begin{pmatrix}
        x' \\
        y' \\
        1 
        \end{pmatrix}
\end{equation}

\item Affine:
\begin{equation}
     \begin{pmatrix}
a_1 & a_2 &  t_x \\
a_3 & a_4 & t_y \\
0 & 0 & 1
    \end{pmatrix}
    \begin{pmatrix}
        x \\
        y \\
        1 
        \end{pmatrix}
        =
        \begin{pmatrix}
        x' \\
        y' \\
        1 
        \end{pmatrix}
\end{equation}

\item Projective:

  \begin{equation}
  \begin{pmatrix}
h_1 & h_2 &  h_3 \\
h_4 & h_5 & h_6 \\
h_7 & h_8 & 1
\end{pmatrix}
\begin{pmatrix}
        x \\
        y \\
        1 
        \end{pmatrix}
        =
        \begin{pmatrix}
        u \\
        v \\
        w 
        \end{pmatrix}
\end{equation}

   \begin{equation}
   x'=\frac{u}{w}\ \  , \ \  y'=\frac{v}{w}
    \end{equation}
    \end{itemize}
\item  What is the number of degrees of freedom in these transformations?

\textbf{Solution:}
\begin{itemize}
    \item Translation: DoF=2
    \item Euclidean: DoF= 3
    \item Similarity: DoF=4
    \item Affine: DoF=6
    \item Projective: DoF=8
\end{itemize}
\item  Why is the number of degrees of freedom in a projective transformation less than the number of elements in a $3\times3$ matrix?

\textbf{Solution:}
        Let consider the following $3 \times 3$ matrix:
        \begin{equation}
        \begin{pmatrix}
        a_{1,1} & a_{1,2} & a_{1,3} \\
        a_{2,1} & a_{2,2} & a_{2,3} \\
        a_{3,1} & a_{3,2} & a_{3,3}
        \end{pmatrix}
        \end{equation}
        Now we apply this matrix on our homogeneous $3$-vector.
        \begin{equation}
        k'\begin{pmatrix}
        x' \\
        y' \\
        1
        \end{pmatrix} = \begin{pmatrix}
        a_{1,1} & a_{1,2} & a_{1,3} \\
        a_{2,1} & a_{2,2} & a_{2,3} \\
        a_{3,1} & a_{3,2} & a_{3,3}
        \end{pmatrix} k \begin{pmatrix}
        x \\
        y \\
        1
        \end{pmatrix}
        \end{equation}
        $k'$ and $k$ are just homogeneous scaling factors that are pulled outside of the homogeneous vectors. Therefore, we have
         \begin{equation}
        \frac{k'}{k}\begin{pmatrix}
        x' \\
        y' \\
        1
        \end{pmatrix} = \begin{pmatrix}
        \frac{a_{1,1}}{a_{3,3}} & \frac{a_{1,2}}{a_{3,3}} & \frac{a_{1,3}}{a_{3,3}} \\
        \frac{a_{2,1}}{a_{3,3}} & \frac{a_{2,2}}{a_{3,3}} & \frac{a_{2,3}}{a_{3,3}} \\
        \frac{a_{3,1}}{a_{3,3}} & \frac{a_{3,2}}{a_{3,3}} & 1
        \end{pmatrix}\begin{pmatrix}
        x \\
        y \\
        1
        \end{pmatrix}
        \end{equation}
        Now take $w = \frac{k'}{k}$, and we have:
        \begin{equation}
        \begin{pmatrix}
        u \\
        v \\
        w
        \end{pmatrix} = \begin{pmatrix}
        \frac{a_{1,1}}{a_{3,3}} & \frac{a_{1,2}}{a_{3,3}} & \frac{a_{1,3}}{a_{3,3}} \\
        \frac{a_{2,1}}{a_{3,3}} & \frac{a_{2,2}}{a_{3,3}} & \frac{a_{2,3}}{a_{3,3}} \\
        \frac{a_{3,1}}{a_{3,3}} & \frac{a_{3,2}}{a_{3,3}} & 1
        \end{pmatrix}\begin{pmatrix}
        x \\
        y \\
        1
        \end{pmatrix}
        \end{equation}
        So it is right that our matrix had nine elements at the begining, but now only eight of them are independent. As such, it follows that a projective transformation has eight degrees of freedom which is less than nine.
\end{enumerate}
\item Planar projective transformation\\
The equation of a line on a plane, $ax + by + c = 0$, can be written as $l^Tx= 0$, where
$l = [a\,b\,c]^T$
and $x$ are homogeneous coordinates for lines and points, respectively. Under a planar projective transformation, represented with an invertible $3\times3$ matrix $H$, points transform as
$$x' = Hx$$
    \begin{enumerate}
        \item Given the matrix $H$ for transforming points, as defined above, define the line transformation (i.e. transformation that gives $l'$ which is a transformed version of $l$).
        
        \textbf{Solution:} It is clear that because $H$ is invertible, there is a matrix $H^{-1}$. It is clear that $l^Tx = x^Tl=0$ gives us the equation $l^TH^{-1}Hx = 0$. This means that all of the point $x'= Hx$ lie on the line $l'^T=l^TH^{-1}$. Thus, the transformed version of $l$ is as follows:
        $$l'^T = l^TH^{-1}$$
        \item A projective invariant is a quantity which does not change its value in the transformation. Using the transformation rules for points and lines, show that two lines, $l_1$, $l_2$, and
two points, $x_1$, $x_2$, not lying on the lines have the following invariant under projective
transformation: 

\textbf{Solution:} Consider transformed version of lines and points as below:
        \begin{gather}
              l'^T_1 = {l^T}_1 H^{-1}\\
              l'^T_2 = {l^T}_2 H^{-1}\\
              x'_1 = Hx_1\\
              x'_2 = Hx_2
        \end{gather}
        Now we check the invariant function after the transformation:
        \begin{equation}
            I' = \frac{(l'^T_1x'_1)(l'^T_2x'_2)}{(l'^T_1x'_2)(l'^T_2x'_1)} = \frac{({l^T}_1 H^{-1}Hx_1)({l^T}_2 H^{-1}Hx_2)}{({l^T}_1 H^{-1}Hx_2)({l^T}_2 H^{-1}Hx_1)} = \frac{(l^T_1x_1)(l^T_2x_2)}{(l^T_1x_2)(l^T_2x_1)} = I
        \end{equation}
        So we have the above invariant under projective transformation, and also projective invariants defined via homogeneous coordinates must be invariant also to arbitrary scaling of the homogeneous coordinate vectors with a non-zero scaling factor, so if we use fewer number of lines of number of points, this will not be an invariant, because they might be an extra factor $k$ stands in numerator or denominator.
     \end{enumerate}
\end{enumerate}
\end{document}
